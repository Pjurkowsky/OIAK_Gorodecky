\documentclass[11pt]{article}
\usepackage{graphicx}
\usepackage[mathletters]{ucs}

\usepackage{float}
\usepackage[utf8x]{inputenc}
\usepackage[polish]{babel}
\usepackage[T1]{fontenc}
\usepackage{titlesec}
\usepackage{array}
\usepackage{multirow}
\usepackage{amsmath}
\renewcommand\*{\cdot}
\setcounter{secnumdepth}{4}
\titleformat{\paragraph}
{\normalfont\normalsize\bfseries}{\theparagraph}{1em}{}
\titlespacing*{\paragraph}
{0pt}{3.25ex plus 1ex minus .2ex}{1.5ex plus .2ex}
\graphicspath{ {images/} }
\newcolumntype{L}[1]{>{\raggedright\let\newline\\\arraybackslash\hspace{0pt}}m{#1}}
\newcolumntype{C}[1]{>{\centering\let\newline\\\arraybackslash\hspace{0pt}}m{#1}}
\newcolumntype{R}[1]{>{\raggedleft\let\newline\\\arraybackslash\hspace{0pt}}m{#1}}
\begin{document}
\title{Organizacja i Architektura komputerów}
\author{Łukasz Wdowiak}
\author{Damian Jabłoński}
\begin{titlepage}
    \begin{center}


        \vspace*{-3cm}

        \includegraphics[width=14cm]{images/image.png}

        \vspace*{2cm}
        \huge
        \textbf{Opracowanie publikacji Danila Gorodocky’ego i Tiziano Villa’y - sumatory modularne}

        \vspace{0.5cm}


        \vspace{1.5cm}

        \textbf{Łukasz Wdowiak 264026} \\
        \textbf{Damian Jabłoński 264025}
        \vspace{2cm}

        \vfill
        Prowadzący: dr inż. Piotr Patronik \\
        Grupa: K03-47a, Pon 15:15-16:55 TN \\
        \vspace{2cm}
        Wydział Informatyki i Telekomunikacji \\
        Informatyka Techniczna \\
        IV semestr\\




    \end{center}
\end{titlepage}

\tableofcontents

\newpage

\section{Cele projektu}

\section{Przebieg projektu}

\section{Algorytmy}
\subsection{Modulo dla dowolnych X i P - bit po bicie}
\subsubsection{Opis algorytmu}
W artykule autorzy zaproponowali sposób obliczania oparty na następującej reprezentacji:
\begin{align}
    X & =P \cdot Q+R=                                                                                           \\
      & =P \cdot 2^\delta \cdot q_\delta+P \cdot 2^{\delta-1} \cdot q_{\delta-1}+\ldots+P \cdot 2^0 \cdot q_0+R
\end{align}


$X(\bmod P)=R$, gdzie $X=\left(x_\psi, x_{\psi-1}, \ldots, x_1\right)$ i $\delta$ jest określona nierównościa $P \cdot 2^{\delta}<2^\psi-1 \leq P \cdot 2^{\delta+1}$.

Na przykład, $X=\left(x_{10}, x_9, \ldots, x_1\right)$ i $P=21$, przy  $\delta=5$. Wynosi:
$$
    \begin{aligned}
        X=21 \cdot Q+R & =                                                                         \\
                       & =21 \cdot 2^5 \cdot q_5+21 \cdot 2^4 \cdot q_4+21 \cdot 2^3 \cdot q_3+    \\
                       & +21 \cdot 2^2 \cdot q_2+21 \cdot 2^1 \cdot q_1+21 \cdot 2^0 \cdot q_0+R .
    \end{aligned}
$$

Każdy kolejny iloczyn częściowy jest wejściem kolejnego bloku obliczeniowego. R jest wynikiem szóstego bloku oraz resztą z dzielenia przez 21.
Jeśli chcemy policzyć $X(\bmod P)=R$, gdzie $X = 888$, a $P = 21$, to:
$$
    \begin{aligned}
         & \text { } X_5 \geq 21 \cdot 2^5 \text {, } 888 \geq 672 \text {,  } X_4=888-21 \cdot 2^5=216 \text {; } \\
         & \text { } X_4<21 \cdot 2^4 \text {,  } 216<336 \text {,  } X_3=216 ;                                    \\
         & \text { } X_3 \geq 21 \cdot 2^3 \text {,  } 216 \geq 168 \text {,  } X_2=216-21 \cdot 2^3=48 \text {; } \\
         & \text {  } X_2<21 \cdot 2^2 \text {,  } 48<84 \text {,  } X_1=48 ;                                      \\
         & \text {  } X_1 \geq 21 \cdot 2^1 \text {,  } 48 \geq 42 \text {,  } X_0=48-21 \cdot 2^1=6 \text {; }    \\
         & \text { } X_0<21 \cdot 2^0 \text {, } 6<21 \text {,  } R=6 \text {. }
    \end{aligned}
$$

W pierwszym kroku porównujemy X z $21 \cdot 2^5$.
Następnie odejmujemy $21 \cdot 2^5$ od X i otrzymujemy $X_4=216$.
W kolejnym kroku porównujemy $X_4$ z $21 \cdot 2^4$ i otrzymujemy $X_3=216$.
Następnie odejmujemy $21 \cdot 2^3$ od $X_3$ i otrzymujemy $X_3=48$.
W kolejnym kroku porównujemy $X_2$ z $21 \cdot 2^2$ i otrzymujemy $X_1=48$.
Następnie odejmujemy $21 \cdot 2^1$ od $X_1$ i otrzymujemy $X_0=6$.
W kolejnym kroku porównujemy $X_0$ z $21 \cdot 2^0$ i otrzymujemy $R=6$.


Jak widać powyżej jest to prosta operacja odejmowania i porównywania, która jest wykonywana w pętli.
\subsubsection{Implementacja algorytmu}
Algorytm został zaimplementowany za pomocą trzech funkcji.
\begin{itemize}
    \item \textbf{calc\_length} - oblicza długość binarną liczby `number` poprzez przesuwanie jej bitów w prawo i zliczanie ilości przejść, zwracając ostateczną długość.
    \item \textbf{get\_delta} - funkcja obliczająca $\delta$ na podstawie parametrów l i P, sprawdzając warunek związanym z potęgami dwójki.
    \item \textbf{mod\_bit\_by\_bit} - funkcja obliczająca resztę z dzielenia - w pętli obliczane są kolejne wartości X.
          Jeżeli $X_i \geq P \cdot 2^i$, to $X_{i-1}=X_i-P \cdot 2^i$, w przeciwnym wypadku $X_{i-1}=X_i$.
          Pętla kończy się, gdy wartość $\delta$ wynosi 0 ,a funkcja zwraca $R$ jako reszte z dzielenia.
\end{itemize}
\begin{thebibliography}{9}
    \bibitem{texbook}
\end{thebibliography}
\end{document}