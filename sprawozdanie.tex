\documentclass[11pt]{article}
\usepackage{graphicx}
\usepackage[mathletters]{ucs}
\usepackage{url}
\usepackage{float}
\usepackage[utf8x]{inputenc}
\usepackage[polish]{babel}
\usepackage[T1]{fontenc}
\usepackage{titlesec}
\usepackage{array}
\usepackage{multirow}
\usepackage{amsmath}
\usepackage{pythonhighlight}
\renewcommand\*{\cdot}
\setcounter{secnumdepth}{4}
\titleformat{\paragraph}
{\normalfont\normalsize\bfseries}{\theparagraph}{1em}{}
\titlespacing*{\paragraph}
{0pt}{3.25ex plus 1ex minus .2ex}{1.5ex plus .2ex}
\graphicspath{ {images/} }
\newcolumntype{L}[1]{>{\raggedright\let\newline\\\arraybackslash\hspace{0pt}}m{#1}}
\newcolumntype{C}[1]{>{\centering\let\newline\\\arraybackslash\hspace{0pt}}m{#1}}
\newcolumntype{R}[1]{>{\raggedleft\let\newline\\\arraybackslash\hspace{0pt}}m{#1}}
\begin{document}
\title{Organizacja i Architektura komputerów}
\author{Łukasz Wdowiak}
\author{Damian Jabłoński}
\begin{titlepage}
    \begin{center}


        \vspace*{-3cm}

        \includegraphics[width=14cm]{images/image.png}

        \vspace*{2cm}
        \huge
        \textbf{Opracowanie publikacji Danila Gorodocky’ego i Tiziano Villa’y - sumatory modularne}

        \vspace{0.5cm}


        \vspace{1.5cm}

        \textbf{Łukasz Wdowiak 264026} \\
        \textbf{Damian Jabłoński 264025}
        \vspace{2cm}

        \vfill
        Prowadzący: dr inż. Piotr Patronik \\
        Grupa: K03-47a, Pon 15:15-16:55 TN \\
        \vspace{2cm}
        Wydział Informatyki i Telekomunikacji \\
        Informatyka Techniczna \\
        IV semestr\\




    \end{center}
\end{titlepage}

\tableofcontents

\newpage

\section{Cele projektu}
Celem projektu jest analiza oraz implementacja poszczególnych algorytmów związanych z arytmetyką modularną zaprezentowanych w artykule naukowym \textit{Efficient Hardware Operations for the Residue Number System by Boolean Minimization} autorstwa Dana Gorodecky i Tomera Villi. W ramach projektu zostaną zaimplementowane oraz wytłumaczone algorytmy:
\begin{itemize}
    \item Modulo dla dowolnych X i P - bit po bicie
    \item Algorytm mnożenia modułowego
\end{itemize}
\section{Założenia projektu}
Nasz projekt powinien skupiać się na implementacji oraz analizie algorytmów związanych z arytmetyką modularną. W ramach projektu powinny zostać zaimplementowane wcześniej wymienione algorytmy.
Algorytmy implementowane będą w języku Python.
\section{Algorytmy}
Autorzy artykułu zaproponowali kilka algorytmów związanych z arytmetyką modularną. W ramach projektu skupiliśmy się na kilku z nich.
\subsection{Modulo dla dowolnych X i P - bit po bicie}
Algorytm ten pozwala na  $X(\bmod P)$ z dowolnych liczb. Jego głowna cechą jest to, że redukujemy liczbę X bit po bicie, aż dojdziemy do reszty z dzielenia.
\subsubsection{Opis algorytmu}
W artykule autorzy zaproponowali sposób obliczania oparty na następującej reprezentacji:
\begin{align}
    X & =P \cdot Q+R=                                                                                           \\
      & =P \cdot 2^\delta \cdot q_\delta+P \cdot 2^{\delta-1} \cdot q_{\delta-1}+\ldots+P \cdot 2^0 \cdot q_0+R
\end{align}


$X(\bmod P)=R$, gdzie $X=\left(x_\psi, x_{\psi-1}, \ldots, x_1\right)$ i $\delta$ jest określona nierównościa $P \cdot 2^{\delta}<2^\psi-1 \leq P \cdot 2^{\delta+1}$.

Na przykład, $X=\left(x_{10}, x_9, \ldots, x_1\right)$ i $P=21$, przy  $\delta=5$. Wynosi:
$$
    \begin{aligned}
        X=21 \cdot Q+R & =                                                                         \\
                       & =21 \cdot 2^5 \cdot q_5+21 \cdot 2^4 \cdot q_4+21 \cdot 2^3 \cdot q_3+    \\
                       & +21 \cdot 2^2 \cdot q_2+21 \cdot 2^1 \cdot q_1+21 \cdot 2^0 \cdot q_0+R .
    \end{aligned}
$$

Każdy kolejny iloczyn częściowy jest wejściem kolejnego bloku obliczeniowego. R jest wynikiem szóstego bloku oraz resztą z dzielenia przez 21.
Jeśli chcemy policzyć $X(\bmod P)=R$, gdzie $X = 888$, a $P = 21$, to:
$$
    \begin{aligned}
         & \text { } X_5 \geq 21 \cdot 2^5 \text {, } 888 \geq 672 \text {,  } X_4=888-21 \cdot 2^5=216 \text {; } \\
         & \text { } X_4<21 \cdot 2^4 \text {,  } 216<336 \text {,  } X_3=216 ;                                    \\
         & \text { } X_3 \geq 21 \cdot 2^3 \text {,  } 216 \geq 168 \text {,  } X_2=216-21 \cdot 2^3=48 \text {; } \\
         & \text {  } X_2<21 \cdot 2^2 \text {,  } 48<84 \text {,  } X_1=48 ;                                      \\
         & \text {  } X_1 \geq 21 \cdot 2^1 \text {,  } 48 \geq 42 \text {,  } X_0=48-21 \cdot 2^1=6 \text {; }    \\
         & \text { } X_0<21 \cdot 2^0 \text {, } 6<21 \text {,  } R=6 \text {. }
    \end{aligned}
$$

W pierwszym kroku porównujemy X z $21 \cdot 2^5$.
Następnie odejmujemy $21 \cdot 2^5$ od X i otrzymujemy $X_4=216$.
W kolejnym kroku porównujemy $X_4$ z $21 \cdot 2^4$ i otrzymujemy $X_3=216$.
Następnie odejmujemy $21 \cdot 2^3$ od $X_3$ i otrzymujemy $X_3=48$.
W kolejnym kroku porównujemy $X_2$ z $21 \cdot 2^2$ i otrzymujemy $X_1=48$.
Następnie odejmujemy $21 \cdot 2^1$ od $X_1$ i otrzymujemy $X_0=6$.
W kolejnym kroku porównujemy $X_0$ z $21 \cdot 2^0$ i otrzymujemy $R=6$.


Jak widać powyżej jest to prosta operacja odejmowania i porównywania, która jest wykonywana w pętli.
\subsubsection{Implementacja algorytmu}
Algorytm został zaimplementowany za pomocą trzech funkcji.
\begin{itemize}
    \item \textbf{calc\_length} - oblicza długość binarną liczby `number` poprzez przesuwanie jej bitów w prawo i zliczanie ilości przejść, zwracając ostateczną długość.
    \item \textbf{get\_delta} - funkcja obliczająca $\delta$ na podstawie parametrów l i P, sprawdzając warunek związanym z potęgami dwójki.
    \item \textbf{mod\_bit\_by\_bit} - funkcja obliczająca resztę z dzielenia - w pętli obliczane są kolejne wartości X.
          Jeżeli $X_i \geq P \cdot 2^i$, to $X_{i-1}=X_i-P \cdot 2^i$, w przeciwnym wypadku $X_{i-1}=X_i$.
          Pętla kończy się, gdy wartość $\delta$ wynosi 0 ,a funkcja zwraca $R$ jako reszte z dzielenia.
\end{itemize}


\subsection{Algorytm mnożenia modułowego}
Algorytm ten pozwala na mnożenie liczb w systemie resztowym. Jego główną cechą jest to, że dzielimy liczby na subwektory, a następnie mnożymy je w sposób opisany poniżej.
\subsubsection{Opis algorytmu}
Autorzy artykułu zaproponowali algorytm mnożenia modułowego, który pozwala policzyć $A \cdot B=R(\bmod P)$, gdzie
$A=\left(A_\delta, A_{\delta-1}, \ldots, A_1\right), B=\left(B_\delta, B_{\delta-1}, \ldots, B_1\right)$  $A_\delta$ oraz $B_\delta$  oznaczają najbardziej znaczące bity liczb A i B, a $\delta$ jest długością słow binarnych z których się składaja.
Np. $A=13$ i $B=14$ to $A=\left(1, 1, 0, 1\right)$ i $B=\left(1, 1, 1, 0\right)$, a $\delta=2$ to $A_2 =\left(1, 1\right)$  oraz $B_2  =\left(1, 1\right)$. Staramy się dzielić liczby wejściowe na dwu, trzy lub czterobitowe  subwektory.
Odpowiednie pary subwektorów mnożymy używając poniższego wzoru:
$$
    R=\sum_{i=1}^\delta \sum_{j=1}^\delta A_i \cdot B_j \cdot\left(2^{m-(i+j-2)-3}(\bmod P)\right)=S_{-} \text {temp }
$$

$S_{temp}$ nie może przekraczać  $ 2^{3 \cdot \delta+2}$
\newpage
Przykład wykorzystania algorytmu:


Wybieramy dwie 6-bitowe liczby $A=45$ and $B=15$ oraz $P=47$. Dzielimy je na 3-bitowe subwectory. Oznacz to, że $\delta = 2$
\\
$A_1=\left(1, 0, 1\right) = 5$
$B_1=\left(1,1,1\right) = 7$
$A_2=\left(1, 0, 1\right) = 7$
$B_2=\left(1\right) = 1$
$A \cdot B=S(\bmod 47)$\\

$S_{temp} = A_1 \cdot B_1(\bmod 47)+A_1 \cdot B_2 \cdot 2^3(\bmod 47)+A_2 \cdot B_1 \cdot 2^3(\bmod 47)+$ $A_2 \cdot B_2 \cdot 2^6(\bmod 47) =5 \cdot 7(\bmod 47)+$ $5 \cdot 1 \cdot 2^3(\bmod 47)+5 \cdot 7 \cdot 2^3(\bmod 47)+5 \cdot 1 \cdot 2^6(\bmod 47)=35(\bmod 47)+40(\bmod 47)+$ $45(\bmod 47)+38(\bmod 47)=158$
\\


$158 >= 2^{3 \cdot 2+2} = 128$, co oznacza, że musimy wykonać kolejna iteracje aby zmniejszyć $S_{temp}$.
\\


$S_{temp} = 158 = \left(1,0,0,1,1,1,1,0\right)$
\\

$S_{temp1} = \left(1,1,0 \right)$
$S_{temp2} = \left(0,1,1 \right)$
$S_{temp3} = \left(1,0 \right)$
\\

$S_{temp} =6+3 \cdot 2^3(\bmod 47)+2 \cdot 2^6(\bmod 47)=6+24+34=64$.
\\

$64 <=  128$, więc

$S(\bmod 47) = 64(\bmod 47) = 17$


\subsubsection{Implementacja algorytmu}
Algorytm został zaimplementowany za pomoc¡ pięciu funkcji.
\begin{itemize}
    \item \textbf{getBitCount} - oblicza długość binarną liczby `X` poprzez przesuwanie jej bitów w prawo i zliczanie ilości przejść, zwracając ostateczną długość.
    \item \textbf{createSubvectors} - funkcja dzieli liczbę `X` na subwektory o długości `subvectorBitCount` i zwraca listę subwektorów.
    \item  \textbf{printSubvectors} - funkcja wypisuje subwektory z listy `subvectors`.
    \item \textbf{subvectorToDec} - funkcja zamienia subwektor na liczbę dziesiętną.
    \item   \textbf{getMultResult} - funkcja oblicza $A \cdot B(\bmod P)$, gdzie $A$ i $B$ są liczbami dziesiętnymi, a $P$ jest liczbą modulo.
\end{itemize}

\section{Wnioski}
Artykuł naukowy skupiał się na kilku sposobach rozwiązania problemu obliczenia modulo z dużej liczby.
Opisane przez nas algorytmy są tylko częscią z nich.
Pierwszy algorytm stosował podejście bit po bice, a drugi dzielił liczby na subwektory.
Podczas implementacji algorytmów w języku Python napotkaliśmy na kilka problemów, które zgrabnie rozwikłaliśmy.
\begin{thebibliography}{9}
    \bibitem{texbook}
    D. Gorodecky and T. Villa, ”Efficient Hardware Operations for the Residue Number
    System by Boolean Minimization”, Advanced Boolean Techniques, Minsk, Belarus,
    January, 2020, p. 237-258
\end{thebibliography}

\newpage
\section{Kod źródłowy}
\subsection{bit\_by\_bit.py}
\begin{python}
    import math


    def calc_length(number):
    length = 0
    while number != 0:
    number >>= 1
    length += 1
    return length


    def get_delta(l, P):
    delta = 0
    while P * math.pow(2, delta) < math.pow(2, l) - 1:
    delta += 1
    return delta


    def mod_bit_by_bit(X, P, delta):
    while delta >= 0:
    if X >= P * math.pow(2, delta):
    X -= P * math.pow(2, delta)
    print("X" + str(delta) + " : ", X)
    delta -= 1
    return X


    def bit_by_bit(X, P):
    l = calc_length(X)
    delta = get_delta(l, P)
    R = mod_bit_by_bit(X, P, delta)
    print("Ilosc bitow: ", l)
    print("Ile iteracji: ", delta)
    print("Reszta: ", R)


    if __name__ == "__main__":
    X = 888
    P = 21
    bit_by_bit(X, P)
\end{python}
\newpage
\subsection{modular\_multiplication.py}


\begin{python}

    import math


    def getBitCount(X):
    if X == 0:
    return 1

    count = 0
    while X > 0:

    count += 1
    X >>= 1
    return count


    def createSubvectors(X, subvectorBitCount, subvectorCount):

    subvectors = []
    for _ in range(subvectorCount):
    subvector = [0] * subvectorBitCount
    subvectors.append(subvector)

    for i in range(subvectorCount):
    for j in range(subvectorBitCount - 1, -1, -1):
    subvectors[i][j] = X % 2
    X //= 2
    return subvectors


    def printSubvectors(subvectors):

    for subvector in subvectors:
    print("[ ", end="")
    for bit in subvector:
    print(bit, end=" ")
    print("]", end=" ")
    print()


    def subvectorToDec(subvector):

    dec = 0
    for i in range(len(subvector)):
    dec += subvector[i] * int(math.pow(2, len(subvector) - i - 1))
    return dec


    def getMultResult(A, B, P):

    bitsOfbiggerNumber = getBitCount(max(A, B))
    subvectorBitCount = 3  # r
    subvectorCount = math.ceil(bitsOfbiggerNumber / subvectorBitCount)  # delta
    print(bitsOfbiggerNumber, subvectorBitCount)

    subvectorsA = createSubvectors(A, subvectorBitCount, subvectorCount)
    printSubvectors(subvectorsA)

    subvectorsB = createSubvectors(B, subvectorBitCount, subvectorCount)
    printSubvectors(subvectorsB)

    result = 0

    for i in range(len(subvectorsA)):
    for j in range(len(subvectorsB)):

    A_iter = subvectorToDec(subvectorsA[i])
    B_iter = subvectorToDec(subvectorsB[j])

    S_temp = A_iter * B_iter * int(math.pow(2,
    ((i + 1) + (j + 1) - 2) * subvectorBitCount)) % P

    result += S_temp

    print(result)

    range_ = int(math.pow(2, 3 * subvectorCount + subvectorCount))

    rangeBitCount = getBitCount(range_)
    resultSubvectorCount = math.ceil(rangeBitCount / subvectorBitCount)
    resultSubvectors = createSubvectors(result, subvectorBitCount,
    resultSubvectorCount)
    printSubvectors(resultSubvectors)

    power = 0
    ans = 0
    for i in range(subvectorCount + 1):
    ans_temp = subvectorToDec(
    resultSubvectors[i]) * int(math.pow(2, power)) % P
    power += subvectorBitCount
    ans += ans_temp
    if ans > P:
    ans -= P
    print(ans)


    A = 45
    B = 15
    P = 47

    getMultResult(A, B, P)


\end{python}

\end{document}